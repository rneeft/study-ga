\chapter{Related work}

\todo{make better introduction for related work. }

\subsection{What is a change?} \label{sec:what-is-change}

Not many research papers could be found that define the word change. A clear definition of the word \textit{change} is found in a psychology paper by Rensink \cite{rensink2002change}.

\begin{quote}
    "The word change generally refers to a transformation or modification of something over time. As such, this notion presumes a nonchanging substrate on which changes are imposed. More precisely, change is defined here [the paper] as the transformation over time of a well-defined, enduring structure." (Rensink, 2002, p. 248).
\end{quote}

The definition of the word change can be translated to the language of \testar. The "well-defined, enduring structure" refers to the GUI of the system under test, whereas the "transformation over time" refers to the various version of the SUT. The "something" refers to a part of the GUI that can change like: Widget tree, the state or an action. 

Besides defining the word change, Rensink also describes the differences between changes and differences. A change is a transformation of the same 'something', whether a difference lacks the property of the same something. For example, what are the differences between two distinct SUT, or the same SUT but in two distinct environments or Internet Browsers? A limit to the change detection algorithm is that the SUT needs to be executed in the same environment; otherwise, false-positive changes can occur.

This distinction of definitions between change and differences give the first limit to this research paper's proposed change detection algorithm. The change detection algorithm will give the changes of a SUT, between versions, in the same environment since a different environment can influence the model and therefore influence the outcome of the change detection.

\subsection{What change detection?}


\section{detecting changed by model comp} 

model 

\section{The Murphy tools} \label{sec:murphy-tools}

F-Secure, a security software company, based in Finland, developed the Murphy tool \cite{aho2013industrial}. With the Murphy tool, it was possible to automatically extract models from the GUI and use them for GUI testing. The goal of the Murphy tool is to find as many states of a GUI as possible. Users can customise the Murphy scripts to direct the tool to particular cases. 

Like \testar, Murphy creates a directed graph of the GUI under test, however they utilise difference approaches in how they create the graph. \testar is a platform-independent but still need platform-dependent api's (Windows Automation API for Windows application, Chromedriver for website) to "read" what is displayed on the screen. Murphy on the other hand is a multi-platform tool since it does not need platform-dependent api's but used screenshots to create the graph.

To "read" the GUI Murphy uses various \emph{drivers}.  A driver recognises elements and windows of the GUI under test. One of the drivers uses the 'tab' key to enumerate UI elements. 

While Murphy is crawling the GUI it is making screenshots of the GUI. Besides that, the screenshots represent the state in the graph; they are used to detect changes between different versions of the SUT. 



GUI driver 

Reference testing GUITAR creates model of current version -> generate test cases... when TC fails -> test failure. 
new features are not failing test cases. 

Browser comparison -> ??

How to compare graphs and visualise them. 

Change visualisation

no much research has been done finding changed in inferred mdoels. There is however a research field that come close to what the research is try to achive, namely cross browser testen. Insted of finding changes between two versions of the same application, with cross browser testing it tries to find differences of the same version but on different browsers.




