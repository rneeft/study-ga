\chapter{Related work} \label{chapter:related-work}

This chapter covers published research on model abstraction and change detection. Section \ref{sec:what-is-change} will give an definition on what change (detection) is and section \ref{sec:change-detection-research} discusses other approaches of dynamic analysis and change detection. 

\section{What is a change \& change detection?} \label{sec:what-is-change}

Not many research papers could be found that define the word change. A clear definition of the word \textit{change} is found in a psychology paper by Rensink \cite{rensink2002change}.

\begin{quote}
    "The word change generally refers to a transformation or modification of something over time. As such, this notion presumes a nonchanging substrate on which changes are imposed. More precisely, change is defined here [the paper] as the transformation over time of a well-defined, enduring structure." (Rensink, 2002, p. 248).
\end{quote}

The definition of the word change can be translated to the context of \testar. The "well-defined, enduring structure" refers to the GUI of the system under test, whereas the "transformation over time" refers to the various version of that SUT. The "something" refers to a part of the GUI that can change, like: Widget tree, a state or actions. 

Besides defining the word change, Rensink also describes the differences between changes and differences. A change is a transformation of the same 'something', whether a difference lacks the property of the same something. For example, what are the differences between two distinct SUT or the same SUT but in two distinct environments or Internet Browsers?

\section{Change detection in previously published research} \label{sec:change-detection-research}

In theory, executing rudimentary change detection with capture \& replay (CR) application is possible. With CR, the tester works with the application under test to record test cases. The tester can replay their recorded test case in the following application version. If the test cases pass, that part of the application under test is not changed. If the test cases fail, something has changed \cite{VosAho2021}.

Despite using existing test cases being an easy way to ensure the AUT still has the same behaviour as in an earlier version, using them cannot be used to find new features. It is, of course, impossible to record new features in previous versions \cite{VosAho2021}. Many applications adopted the use of generated test cases to find an early form of change detection \cite{aho2019automated}. Three of the those tools are discussed in sections \ref{sec:guitar}, \ref{sec:guitam} and \ref{sec:gui-driver}.

The section \ref{sec:murphy-tool} discusses the Murphy tool, which uses a state-of-the-art approach in which two consequent applications are modelled, and the two models are compared to find changes \cite{aho2019automated}.

\subsection{Guitar (Bao N. Nguyen, Bryan Robbins, Ishan Banerjee \& Atif Memon)} \label{sec:guitar}

\emph{Guitar} \cite{nguyen2014guitar} is a suite of tools, each having a distinct purpose. The ripper tool is responsible for generating a model based on the GUI of the AUT, named GUI Tree. The GUI Tree contains all the widgets and properties for each GUI window. The output of the ripper is used by the Graph Converter that converts the GUI Tree into an Event-Flow Graph (EFG). The EFG is used by the Test Case Generator tool to create executable test cases, which the Replayer tool uses on the AUT. With a plugin for the replayer, it can be used as a monitor tool to record the AUT behaviour and compare it with a previously recorded state to report changes.

\subsection{GuiTam (Yuan Miao \& Xuebing Yang)} \label{sec:guitam}
\emph{GuiTam} \cite{miao2010fsm} (GUI Test Automation Model) is a finite state machine (FSM) based on the GUI. By traversing all the states in FSM, test cases can be generated. The test cases are executed on a modified version of the AUT. During test execution, state information of the new version can be reflected in the old version of the application. When differences are found, they may reflect defects. 

\subsection{GUI Driver (Pekka Aho, Nadja Menz, Tomi Räty \& Ina Schieferdecker)}  \label{sec:gui-driver}
\emph{GUI Driver} \cite{aho2011automated} creates a model using the structure tree nodes of the GUI components and their properties. An MBT application \cite{mbt-tigris-org} outside of GUI Drivers creates test cases based on the models. GUI Drivers use the output of the MBT application to execute test cases. During test execution, GUI Driver compares the state of the old version with the current version to determine whether the expected state is reached. A report is generated with the results of the test run. 

\subsection{Murphy tool (F-Secure)} \label{sec:murphy-tool}
WithSecure, previously known as F-Secure, a security software company based in Finland, developed a tool named Murphy \cite{aho2013industrial}. With Murphy, it is possible to automatically extract models from the GUI and use them for GUI testing. The goal of the Murphy tool is to find as many states (nodes) of a GUI as possible. Users can customise the behaviour of Murphy with scripts to direct the tool to particular cases. 

Like \testar, Murphy is analysing the GUI as if it is a user. A directed graph of the GUI with its state is created during the analysis. Even though \testar and Murphy are creating a graph about the GUI, the method of "reading" the GUI is different. The graph is visualised by a screenshot of each state's GUI \cite{aho2014murphy}.

\testar is platform-independent but still needs platform-dependent APIs (e.g. Windows Automation API for Windows application, Chromedriver for website) to read what is displayed on the screen. On the other hand, Murphy is a multi-platform tool since it does not need platform-dependent APIs but uses screenshots to create the graph \cite{aho2013industrial}.

\subsubsection{Change detection}
Murphy can be run in a continuous integration process to extract a new model for each new version of the GUI. It can automatically find changes by comparing the models \cite{aho2014murphy}.  

The user can view the changes in a Murphy web UI. Two screenshots in this web interface are visible, accentuating the detection change. The test engineer can indicate to Murphy whether the change was as expected or not.