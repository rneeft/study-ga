\chapter{Custom Protocol for experiment application} \label{appendix:protocol-experiment}

\begin{lstlisting}[language=java, basicstyle=\tiny, caption=Protocol for the experiment application, label=code:protocol-experiment]
import org.testar.CodingManager;
import java.util.Set;
import org.testar.RandomActionSelector;
import org.testar.monkey.alayer.*;
import org.testar.monkey.alayer.actions.*;
import org.testar.monkey.alayer.exceptions.ActionBuildException;
import org.testar.protocols.DesktopProtocol;
import org.testar.monkey.alayer.exceptions.NoSuchTagException;
import java.util.*;
import java.util.zip.CRC32;

public class Protocol_desktop_generic_statemodel extends DesktopProtocol {

	@Override
	protected void buildStateActionsIdentifiers(State state, Set<Action> actions) {
		CodingManager.buildIDs(state, actions);
		
		System.out.println("buildStateActionsIdentifiers");
		// Custom the Action AbstractIDCustom identifier
		for(Action a : actions) {
			if(a.get(Tags.OriginWidget) != null) {
				Widget widget = a.get(Tags.OriginWidget);

				// set the identifier to only containing the title and not to include the parent
				String widgetTitle = widget.get(Tags.Title);
				System.out.println("Widget Tag.Title - " + widgetTitle);
				String customIdentifier = CodingManager.ID_PREFIX_ACTION + CodingManager.ID_PREFIX_ABSTRACT_CUSTOM + lowCollisionID(widgetTitle);
				a.set(Tags.AbstractIDCustom, customIdentifier);

				System.out.println("  - " + customIdentifier);
			}
		}
	}

	// ############
	//  IDS CODING
	// ############
	private static String lowCollisionID(String text){ // reduce ID collision probability
		CRC32 crc32 = new CRC32();
		crc32.update(text.getBytes());
		return Integer.toUnsignedString(text.hashCode(), Character.MAX_RADIX) +
			   Integer.toHexString(text.length()) +
			   crc32.getValue();
	}
	/**
	 * This method is used by TESTAR to determine the set of currently available actions.
	 * You can use the SUT's current state, analyze the widgets and their properties to create
	 * a set of sensible actions, such as: "Click every Button which is enabled" etc.
	 * The return value is supposed to be non-null. If the returned set is empty, TESTAR
	 * will stop generation of the current action and continue with the next one.
	 * @param system the SUT
	 * @param state the SUT's current state
	 * @return  a set of actions
	 */
	@Override
	protected Set<Action> deriveActions(SUT system, State state) throws ActionBuildException{

		//The super method returns a ONLY actions for killing unwanted processes if needed, or bringing the SUT to
		//the foreground. You should add all other actions here yourself.
		// These "special" actions are prioritized over the normal GUI actions in selectAction() / preSelectAction().
		Set<Action> actions = super.deriveActions(system,state);


		// Derive left-click actions, click and type actions, and scroll actions from
		// top level (highest Z-index) widgets of the GUI:
		actions = deriveClickTypeScrollActionsFromTopLevelWidgets(actions, system, state);

		if(actions.isEmpty()){
			// If the top level widgets did not have any executable widgets, try all widgets:
//			System.out.println("No actions from top level widgets, changing to all widgets.");
			// Derive left-click actions, click and type actions, and scroll actions from
			// all widgets of the GUI:
			actions = deriveClickTypeScrollActionsFromAllWidgetsOfState(actions, system, state);
		}

		//return the set of derived actions
		return actions;
	}

	/**
	 * Select one of the available actions using an action selection algorithm (for example random action selection)
	 *
	 * @param state the SUT's current state
	 * @param actions the set of derived actions
	 * @return  the selected action (non-null!)
	 */
	@Override
	protected Action selectAction(State state, Set<Action> actions){

		//Call the preSelectAction method from the AbstractProtocol so that, if necessary,
		//unwanted processes are killed and SUT is put into foreground.
		Action retAction = super.selectAction(state, actions);
		if (retAction== null) {
			//if no preSelected actions are needed, then implement your own action selection strategy
			//using the action selector of the state model:
			retAction = stateModelManager.getAbstractActionToExecute(actions);
		}
		if(retAction==null) {
			System.out.println("State model based action selection did not find an action. Using random action selection.");
			// if state model fails, use random (default would call preSelectAction() again, causing double actions HTML report):
			retAction = RandomActionSelector.selectAction(actions);
		}
		return retAction;
	}
}
\end{lstlisting}