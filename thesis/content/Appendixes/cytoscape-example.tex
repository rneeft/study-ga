\chapter{Cytoscape example} \label{appendix:cytoscape-example}

The code in listing \ref{code:cytoscape-example} is used in section \ref{sec:graph-visualisation} to show a graph with Cytoscape in HTML.

\begin{lstlisting}[language=html,basicstyle=\tiny, caption=Graph representation in JSON, label=code:cytoscape-example]
<html>
    <head>
        <title>TESTAR Cytoscape Example</title>
        <script src="https://cdnjs.cloudflare.com/ajax/libs/cytoscape/3.5.2/cytoscape.min.js"></script>
        <style>
            #cy {
                width: 100%;
                height: 100%;
                position: absolute;
                top: 0px;
                left: 0px;
            }
        </style>
    </head>
    <body>
        <div id="cy"></div>
        <script>
            var cy = cytoscape({
              container: document.getElementById('cy'),
              style: [
                {
                    selector: '.AbstractAction',
                    style: {
                        'line-color': '#1c9099',
                        'target-arrow-color': '#1c9099',
                        'label': 'data(id)'
                    }
                },
                {
                    selector: '.AbstractState',
                    style: {
                        'background-color': '#1c9099',
                        'label': 'data(id)'
                    }
                },
              ],
              elements: 
              [
                {
                    "group": "nodes",
                    "data": {
                      "id": "n100" 
                    },
                    "classes":[
                        "AbstractState"
                    ]
                },
                {
                    "group": "nodes",
                    "data": {
                        "id": "n101"
                    },
                    "classes":[
                        "AbstractState"
                    ]
                },
                {
                    "group": "edges",
                    "data": {
                        "id":"e99",
                        "target": "n100",
                        "source": "n101"
                    },
                    "classes" :[
                        "AbstractAction"
                    ]
                }
              ]
            });
          </script>
    </body>
</html>
\end{lstlisting}