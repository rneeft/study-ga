\chapter{Introduction} \label{chapter:introduction}

Regression testing is considered a good practice when testing new software versions before being released to the general public. 
However, the time to market has decreased drastically due to shortened release cycles. As a result, software test teams have less time to test all new software features, let alone try all other elements to prevent unwanted side effects \cite{rapid-release-cycle-issues}.
This thesis provides a change detection algorithm to find changes between two versions of an application GUI. 

This section introduces the thesis discussing the background and context, followed by the research problem, research aims, research questions and why this research is essential.

\section{Background}
The \testar tool solves a significant obstacle when it comes to testing the GUI of an application. With \testar, the tester can automatically start testing the GUI without any upfront scripts. \testar automatically generates and executes test sequences based on elements derived from the GUI \cite{VosAho2021}. In recent master graduation assignments, \testar has been extended with an inferred model generation module \cite{thesisMulders}, and it became easier to integrate \testar in build and release pipelines in a DevOps environment \cite{thesisSlomp}. Those two additions make it easier to run \testar upon each source code integration and retrieve an inferred model from the GUI. Two inferred models were used in a proof-of-concept application \cite{stateDiff}, that shows that change detection can be done. 

Some companies might write down changes to the software in a changelog. However, some changelogs might not be complete and unwanted side effects might be missing entirely. Comparing two inferred models shows all the changes between versions of a SUT, even the unwanted ones.

\section{Research problem}
Besides a proof-of-concept approach, \testar does not have a feature that compares two inferred models to find changes in two versions of a GUI of the application under test. \testar has an option to replay a previously recorded sequence \cite{testar-manual} on a newer version of the application but that only yield removal results because the previously known execution is being executed on a newer version. It cannot find any new features in the GUI \cite{VosAho2021}.

\section{The aim of the research}
The research aim is to design a change detection algorithm to find changes between two inferred models of two versions of the same application and is formulated in the following main research question:

\textbf{\rqMainQuestion}

The main research question is divided into three sub-questions: "\textit{\rqApplicationOutsideTestar}", which research how to set up an analysis web application outside of \testar. 

The second question "\textit{\rqHowMakingChangeDetectionAlgorithm}" will use the new components of \ref{rq:application-outside-testar} to design an algorithm to find changes between two inferred models. 

To visualise the outcome of \ref{rq:how-making-change-detection-algorithm} the third and last research question is formulated as "\textit{\rqHowToVisualiseResult}"

\section{Scope}
The scope of the research is change detection in GUI inferred models created by \testar. Researching how to make models and making significant changes to the creation of models are omitted. However, tweaking the model generation, like adding data that is not saved at the moment, and configuring \testar to create a good model, are in scope.

\section{Contribution}
When change detection is available in \testar, it becomes easy for testers to run the comparison in a continuous integration environment and receive an overview of the changes in an updated software version.

As a side effect, the comparison and visualisation solutions can run outside the context of \testar and be deployed to a docker environment.

\section{Document outline}
The thesis is structured as follows, in chapter \ref{chapter:introduction}, the context, aims, and objectives for the thesis are introduced, together with the limitations for the outcome. Chapter \ref{chapter:background} describes the thesis's background and contains available knowledge that the readers might not know. Chapter \ref{chapter:related-work} covers published work on research on what change is, model abstractions and change detection. Chapter \ref{chapter:research} discussed the questions for this thesis, whereas chapter \ref{chapter:results} shows the results of the questions and how it is implemented. After the results, the outcome is validated with experiments explained in chapter \ref{chapter:validation-experiments}. The last chapter \ref{chapter:disussion-conclusion-future-work} provides a discussion and possible future research topics.